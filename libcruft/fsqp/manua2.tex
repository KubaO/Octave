\section{Examples}
\label{example}
The first problem is borrowed 
from\Lspace \Lcitemark 9\Rcitemark \Rspace{} (Problem 32).
It involves a single objective function, simple bounds on the variables,
nonlinear inequality constraints,
and linear equality constraints.
The objective function $f$ is defined for $x\in R^3$ by
\begin{quote}
\begin{quote}
$f(x)=(x_1+3x_2+x_3)^2+4(x_1-x_2)^2$
\end{quote}
\end{quote}
The constraints are
\begin{quote}
\begin{quote}
   $0 \leq x_i ,~~~~~~~~~~i = 1,\ldots,3$ \\
   $x_1^3-6x_2-4x_3+3 \leq 0\;\;\;\;\;\
                    1-x_1-x_2-x_3 = 0$
\end{quote}
\end{quote}
The feasible initial guess is:~~~$x_0=(0.1,~0.7,~0.2)^T$ ~~with 
corresponding value
of the objective function~~~ $f(x_0)=7.2$. 
The final solution is:~~~$x^*=(0,~0,~1)^T$ ~~with ~~~$f(x^*)=1$.
A suitable main program is as follows.
\begin{quote}
\begin{verbatim}
c
c     problem description
c
      program sampl1
c
      integer iwsize,nwsize,nparam,nf,nineq,neq
      parameter (iwsize=29, nwsize=219)
      parameter (nparam=3, nf=1)
      parameter (nineq=1, neq=1)
      integer iw(iwsize)
      double  precision x(nparam),bl(nparam),bu(nparam),
     *        f(nf+1),g(nineq+neq+1),w(nwsize)
      external obj32,cntr32,grob32,grcn32
c
      integer mode,iprint,miter,nineqn,neqn,inform
      double precision bigbnd,eps,epseqn,udelta
c
      mode=100
      iprint=1
      miter=500
      bigbnd=1.d+10
      eps=1.d-08
      epseqn=0.d0
      udelta=0.d0
c
c     nparam=3
c     nf=1
      nineqn=1
      neqn=0
c     nineq=1
c     neq=1
c
      bl(1)=0.d0
      bl(2)=0.d0
      bl(3)=0.d0
      bu(1)=bigbnd
      bu(2)=bigbnd
      bu(3)=bigbnd
c
c     give the initial value of x
c
      x(1)=0.1d0
      x(2)=0.7d0
      x(3)=0.2d0
c
      call FSQPD(nparam,nf,nineqn,nineq,neqn,neq,mode,iprint,
     *           miter,inform,bigbnd,eps,epseqn,udelta,bl,bu,x,f,g,
     *           iw,iwsize,w,nwsize,obj32,cntr32,grob32,grcn32)
      end
\end{verbatim}
\end{quote}
Following are the subroutines defining the objective and 
constraints and their gradients.
\begin{quote}
\begin{verbatim}
      subroutine obj32(nparam,j,x,fj)
      integer nparam,j
      double precision x(nparam),fj
c
      fj=(x(1)+3.d0*x(2)+x(3))**2+4.d0*(x(1)-x(2))**2
      return
      end
c
      subroutine grob32(nparam,j,x,gradfj,dummy)
      integer nparam,j
      double  precision x(nparam),gradfj(nparam),fa,fb
      external dummy
c    
      fa=2.d0*(x(1)+3.d0*x(2)+x(3))
      fb=8.d0*(x(1)-x(2))
      gradfj(1)=fa+fb
      gradfj(2)=fa*3.d0-fb
      gradfj(3)=fa
      return
      end
c
      subroutine cntr32(nparam,j,x,gj)
      integer nparam,j
      double precision x(nparam),gj
      external dummy
c
      go to (10,20),j
 10   gj=x(1)**3-6.0d0*x(2)-4.0d0*x(3)+3.d0
      return
 20   gj=1.0d0-x(1)-x(2)-x(3)
      return
      end
c
      subroutine grcn32(nparam,j,x,gradgj,dummy)
      integer nparam,j
      double  precision x(nparam),gradgj(nparam)
      external dummy
c
      go to (10,20),j
 10   gradgj(1)=3.d0*x(1)**2
      gradgj(2)=-6.d0
      gradgj(3)=-4.d0
      return
 20   gradgj(1)=-1.d0
      gradgj(2)=-1.d0
      gradgj(3)=-1.d0
      return
      end
\end{verbatim}
\end{quote}
The file containing the user-provided subroutines is 
then compiled together with {\tt fsqpd.f} and {\tt qld.f}.
After running the algorithm on a SUN 4/SPARC station 1, the 
following output is obtained:
\begin{verbatim}

                   FSQP Version 3.2 (Released March 1993)
                        Copyright (c) 1989 --- 1993         
                          J.L. Zhou and A.L. Tits           
                            All Rights Reserved 
              
              
               The given initial point is feasible for inequality
                      constraints and linear equality constraints:
               x                     0.10000000000000E+00
                                     0.70000000000000E+00
                                     0.20000000000000E+00
               objectives            0.72000000000000E+01
               constraints          -0.19990000000000E+01
                                     0.55511151231258E-16


               iteration                                3
               x                    -0.98607613152626E-31
                                     0.00000000000000E+00
                                     0.10000000000000E+01
               objectives            0.10000000000000E+01
               constraints          -0.10000000000000E+01
                                     0.00000000000000E+00
               SCV                   0.00000000000000E+00
               d0norm                0.13945222387368E-30
               ktnorm                0.10609826585190E-29
               ncallf                                   3
               ncallg                                   5


               inform                                   0
               Normal termination: You have obtained a solution !!
\end{verbatim}

Our second example is taken from example 6 
in\Lspace \Lcitemark 10\Rcitemark \Rspace{}. The problem is as follows.
$$\begin{array}{cl}
\min\limits_{x\in R^6} & \max\limits_{i=1,\ldots,163} |f_i(x)| \\
\mbox{s.t.} & -x(1)~~~~~~\,\!~~~~~~~~~~~~~~\;
~~~~~~~~~~~~~~~~~~~~~~~~~~~~~~~~~+s \leq 0 \\
            & ~\,\,\,\!x(1)-x(2)~~~~~~~~~~~~\;
~~~~~~~~~~~~~~~~~~~~~~~~~~~~~~~~+s \leq 0 \\
            & ~~~~~~\,\!~~~~~\,x(2)-x(3)~~~~\;\,
~~~~~~~~~~~~~~~~~~~~~~~~~~~~~~+s \leq 0 \\
            & ~~~~~~~\,\!~~~~~~~~~~~~~~x(3)-x(4)
~~~~~~~~~~~~~~~~~~~~~~~~\;~+s \leq 0 \\
            & ~~~~~~~~\,\!~~~~~~~~~~~~~~~~~~~~~~\,x(4)-x(5)
~~~~~~~~~~~~~~~\;\,+s \leq 0 \\
          & ~~~~~~~~~~~\,\!~~~~~~~~~~~~~~~~~~~~~~~~~~~~\,\,x(5)-x(6)
~~~\;~~~+s \leq 0 \\
       & ~~~~~~~~~~~~~~~\,\!~~~~~~~~~~~~~~~~~~~~~~~~~~~~~~
~~~\,x(6)-3.5+s \leq 0;
\end{array}$$
where
$$\begin{array}{l}
f_i(x)=\frac{1}{15}+\frac{2}{15}(\sum^6_{j=1} 
           \mbox{cos}(2\pi x_j\mbox{sin}\theta _i)
                +\mbox{cos}(7\pi \mbox{sin}\theta _i)), \\
\theta _i=\frac{\pi}{180}(8.5+0.5i),~i=1,\ldots,163,\\
s=0.425.
\end{array}$$
The feasible initial guess is:~$x_0=(0.5,1,1.5,2,2.5,3)^T$ with
the corresponding value of the objective 
function $\max\limits_{i=1,\ldots,163} |f_i(x_0)|=0.22051991555531$. 
A suitable main program is as follows.
\begin{quote}
\begin{verbatim}
c
c     problem description
c
      program sampl2
c
      integer iwsize,nwsize,nparam,nf,nineq,neq
      parameter (iwsize=1029, nwsize=7693)
      parameter (nparam=6, nf=163)
      parameter (nineq=7, neq=0)
      integer iw(iwsize)
      double  precision x(nparam),bl(nparam),bu(nparam),
     *        f(nf+1),g(nineq+neq+1),w(nwsize)
      external objmad,cnmad,grobfd,grcnfd
c
      integer mode,iprint,miter,nineqn,neqn,inform
      double precision bigbnd,eps,udelta
c
      mode=111
      iprint=1
      miter=500
      bigbnd=1.d+10
      eps=1.0d-08
      epseqn=0.d0
      udelta=0.d0
c
c     nparam=6
c     nf=163
      nineqn=0
      neqn=0
c     nineq=7
c     neq=0
c    
      bl(1)=-bigbnd
      bl(2)=-bigbnd
      bl(3)=-bigbnd
      bl(4)=-bigbnd
      bl(5)=-bigbnd
      bl(6)=-bigbnd
      bu(1)=bigbnd
      bu(2)=bigbnd
      bu(3)=bigbnd
      bu(4)=bigbnd
      bu(5)=bigbnd
      bu(6)=bigbnd
c
c     give the initial value of x
c
      x(1)=0.5d0
      x(2)=1.d0
      x(3)=1.5d0
      x(4)=2.d0
      x(5)=2.5d0
      x(6)=3.d0
c
      call FSQPD(nparam,nf,nineqn,nineq,neqn,neq,mode,iprint,     
     *           miter,inform,bigbnd,eps,epseqn,udelta,bl,bu,x,f,g,
     *           iw,iwsize,w,nwsize,objmad,cnmad,grobfd,grcnfd)
      end
      stop
\end{verbatim}
\end{quote}
We choose to compute the gradients of functions by means of
finite difference approximation. Thus only subroutines that
define the objectives and constraints are needed as follows.
\begin{quote}
\begin{verbatim}
      subroutine objmad(nparam,j,x,fj)
      integer nparam,j,i
      double precision x(nparam),theta,pi,fj
c
      pi=3.14159265358979d0
      theta=pi*(8.5d0+dble(j)*0.5d0)/180.d0
      fj=0.d0
      do 10 i=1,6
 10     fj=fj+dcos(2.d0*pi*x(i)*dsin(theta))
      fj=2.d0*(fj+dcos(2.d0*pi*3.5d0*dsin(theta)))/15.d0
     *  +1.d0/15.d0   
      return
      end
c     
      subroutine cnmad(nparam,j,x,gj)
      integer nparam,j
      double precision x(nparam),ss,gj
c
      ss=0.425d0
      goto(10,20,30,40,50,60,70),j
 10   gj=ss-x(1)
      return
 20   gj=ss+x(1)-x(2)
      return
 30   gj=ss+x(2)-x(3)
      return
 40   gj=ss+x(3)-x(4)
      return
 50   gj=ss+x(4)-x(5)
      return
 60   gj=ss+x(5)-x(6)
      return
 70   gj=ss+x(6)-3.5d0
      return
      end
\end{verbatim}
\end{quote}
After running the algorithm on a SUN 4/SPARC station 1,
the following output
is obtained (the results for the set of objectives have been deleted to
save space)
\begin{verbatim}

                    FSQP Version 3.2 (Released March 1993)    
                         Copyright (c) 1989 --- 1993          
                           J.L. Zhou and A.L. Tits    
                             All Rights Reserved              
             
             
              The given initial point is feasible for inequality
                     constraints and linear equality constraints:
                                    0.50000000000000E+00
                                    0.10000000000000E+01
                                    0.15000000000000E+01
                                    0.20000000000000E+01
                                    0.25000000000000E+01
                                    0.30000000000000E+01
              objmax                0.22051986506559E+00
              constraints          -0.75000000000000E-01
                                   -0.75000000000000E-01
                                   -0.75000000000000E-01
                                   -0.75000000000000E-01
                                   -0.75000000000000E-01
                                   -0.75000000000000E-01
                                   -0.75000000000000E-01

                 
              iteration                                7
              x                     0.42500000000000E+00
                                    0.85000000000000E+00
                                    0.12750000000000E+01
                                    0.17000000000000E+01
                                    0.21840763196688E+01
                                    0.28732755096448E+01
              objective max4        0.11421841325221E+00
              objmax                0.11310472749826E+00
              constraints           0.00000000000000E+00
                                    0.00000000000000E+00
                                    0.00000000000000E+00
                                    0.00000000000000E+00
                                   -0.59076319668817E-01
                                   -0.26419918997596E+00
                                   -0.20172449035522E+00
              SCV                   0.00000000000000E+00
              d0norm                0.15662162275640E-09
              ktnorm                0.20564110435030E-10
              ncallf                                1141
             
             
              inform                                   0
              Normal termination: You have obtained a solution !!
\end{verbatim}

Our third example is borrowed 
from\Lspace \Lcitemark 9\Rcitemark \Rspace{} (Problem 71). It involves both
equality and inequality nonlinear constraints and is defined by
$$\begin{array}{cl}
\min\limits_{x\in R^4} & x_1x_4(x_1+x_2+x_3)+x_3\\
\mbox{s.t.} & 1\leq x_i\leq 5,\quad i=1,\ldots,4\\
            & x_1x_2x_3x_4-25\geq 0\\
            & x_1^2+x_2^2+x_3^2+x_4^2-40=0.
\end{array}$$
The feasible initial guess is: $x_0=(1,5,5,1)^T$ with the corresponding
value of the objective function $f(x_0)=16$. A suitable program
that invokes FSQP to solve this problem is given below.
\begin{quote}
\begin{verbatim}
c
c     problem description
c
      program sampl3
c
      integer iwsize,nwsize,nparam,nf,nineq,neq
      parameter (iwsize=33, nwsize=284)
      parameter (nparam=4, nf=1)
      parameter (nineq=1, neq=1)
      integer iw(iwsize)
      double  precision x(nparam),bl(nparam),bu(nparam),f(nf+1),
     *        g(nineq+neq+1),w(nwsize)
      external obj,cntr,gradob,gradcn
c
      integer mode,iprint,miter,neqn,nineqn,inform
      double precision bigbnd,eps,epseqn,udelta
c
      mode=100
      iprint=1
      miter=500
      bigbnd=1.d+10
      eps=1.0d-07
      epseqn=7.d-06
      udelta=0.d0
c
      neqn=1
      nineqn=1
c
      bl(1)=1.d0
      bl(2)=1.d0
      bl(3)=1.d0
      bl(4)=1.d0
      bu(1)=5.d0
      bu(2)=5.d0
      bu(3)=5.d0
      bu(4)=5.d0
c
c     give the initial value of x
c
      x(1)=1.d0
      x(2)=5.d0
      x(3)=5.d0
      x(4)=1.d0
c
      call FSQPD(nparam,nf,nineqn,nineq,neqn,neq,mode,iprint,
     *           miter,inform,bigbnd,eps,epseqn,udelta,bl,bu,x,f,g,
     *           iw,iwsize,w,nwsize,obj,cntr,gradob,gradcn)
      end
\end{verbatim}
\end{quote}
Following are the subroutines that define the objective, constraints
and their gradients.
\begin{quote}
\begin{verbatim}
      subroutine obj(nparam,j,x,fj)
      integer nparam,j
      double precision x(nparam),fj
c
      fj=x(1)*x(4)*(x(1)+x(2)+x(3))+x(3)
      return
      end
c     
      subroutine gradob(nparam,j,x,gradfj,dummy)
      integer nparam,j
      double precision x(nparam),gradfj(nparam)
      external dummy
c  
      gradfj(1)=x(4)*(x(1)+x(2)+x(3))+x(1)*x(4)
      gradfj(2)=x(1)*x(4)
      gradfj(3)=x(1)*x(4)+1.d0
      gradfj(4)=x(1)*(x(1)+x(2)+x(3))
      return
      end
c     
      subroutine cntr(nparam,j,x,gj)
      integer nparam,j
      double precision x(nparam),gj
c
      goto (10,20),j
 10   gj=25.d0-x(1)*x(2)*x(3)*x(4)
      return
 20   gj=x(1)**2+x(2)**2+x(3)**2+x(4)**2-40.d0
      return
      end
c
      subroutine gradcn(nparam,j,x,gradgj,dummy)
      integer nparam,j
      double precision x(nparam),gradgj(nparam)
      external dummy
c
      goto (10,20),j
 10   gradgj(1)=-x(2)*x(3)*x(4)
      gradgj(2)=-x(1)*x(3)*x(4)
      gradgj(3)=-x(1)*x(2)*x(4)
      gradgj(4)=-x(1)*x(2)*x(3)
      return
 20   gradgj(1)=2.d0*x(1)
      gradgj(2)=2.d0*x(2)
      gradgj(3)=2.d0*x(3)
      gradgj(4)=2.d0*x(4)
      return
      end
\end{verbatim}
\end{quote}
After running the algorithm on a SUN 4/SPARC station 1, the following
output is obtained
\begin{verbatim}

                    FSQP Version 3.2 (Released March 1993)    
                         Copyright (c) 1989 --- 1993          
                           J.L. Zhou and A.L. Tits    
                             All Rights Reserved              
             
                          
              The given initial point is feasible for inequality
                     constraints and linear equality constraints:
              x                     0.10000000000000E+01
                                    0.50000000000000E+01
                                    0.50000000000000E+01
                                    0.10000000000000E+01
              objectives            0.16000000000000E+02
              constraints           0.00000000000000E+00
                                   -0.12000000000000E+02
             
             
              iteration                                8
              x                     0.10000000000000E+01
                                    0.47429996518112E+01
                                    0.38211499651796E+01
                                    0.13794082958030E+01
              objectives            0.17014017289158E+02
              constraints          -0.35171865420125E-11
                                   -0.35100811146549E-11
              SCV                   0.35100811146549E-11
              d0norm                0.23956399867788E-07
              ktnorm                0.34009891628142E-07
              ncallf                                   9
              ncallg                                  24


              inform                                   0
              Normal termination: You have obtained a solution !!
\end{verbatim}

\section{Results for Test Problems}
\label{results}
\noindent These results are provided to allow the user to
compare FSQP with his/her favorite code (see 
also\Lspace \Lcitemark 2\Citehyphen 4\Rcitemark \Rspace{}). 
Table 1 contains results 
obtained for some (non-minimax) test problems 
from\Lspace \Lcitemark 9\Rcitemark \Rspace{} (the same initial points
as in\Lspace \Lcitemark 9\Rcitemark \Rspace{} were selected).
{\tt prob} indicates the problem number as in
\Lcitemark 9\Rcitemark \Rspace{}, {\tt nineqn} the number of nonlinear constrai
nts,
{\tt ncallf} the total number of evaluations of the objective function,
{\tt ncallg} the total number of evaluations of the (scalar) nonlinear 
constraint functions, {\tt iter} the total number of iterations, 
{\tt objective} the final value 
of the objective, {\tt ktnorm} the norm of Kuhn-Tucker vector at the 
final iterate, {\tt eps} the norm requirement of the Kuhn-Tucker vector,
{\tt SCV} the sum of feasibility violation of linear constraints (see 
\S~\ref{stopcri}). On each test problem, {\tt eps} was selected 
so as to achieve the same
field precision as in\Lspace \Lcitemark 9\Rcitemark \Rspace{}. 
Whether FSQP-AL (0) or FSQP-NL (1) is used is indicated in column ``B''.

Results obtained on selected minimax problems 
are summarized in Table 2.
Problems {\tt bard}, {\tt davd2}, {\tt f\&r}, {\tt hettich}, 
and {\tt wats} are
from\Lspace \Lcitemark 11\Rcitemark \Rspace{};
{\tt cb2}, {\tt cb3}, {\tt r-s}, {\tt wong} and {\tt colv} are 
from\Lspace \Lcitemark 12\LIcitemark{}; Examples 5.1-5\RIcitemark \Rcitemark \R
space{} 
(the latest test results on problems {\tt bard} down to
{\tt wong} can be found in\Lspace \Lcitemark 13\Rcitemark \Rspace{});
{\tt kiw1} and {\tt kiw4} are from\Lspace \Lcitemark 14\Rcitemark \Rspace{}
(results for {\tt kiw2} and {\tt kiw3} are not reported due to 
data disparity);
{\tt mad1} to {\tt mad8} are from\Lspace \Lcitemark 10\LIcitemark{}, Examples 1
-8\RIcitemark \Rcitemark \Rspace{};
{\tt polk1} to {\tt polk4} are from\Lspace \Lcitemark 15\Rcitemark \Rspace{}.
%for {\tt polk1} to {\tt polk4}. 
Some of these test problems allow one to freely select 
the number of variables;
problems {\tt wats-6} and {\tt wats-20} correspond to 6 and 20 
variables respectively,
and {\tt mad8-10}, {\tt mad8-30} and {\tt mad8-50} to 10, 30 and 50
variables respectively.
All of the above are either
unconstrained or linearly constrained minimax problems. Unable to find
nonlinearly constrained minimax test problems in the literature, we 
constructed problems {\tt p43m} through {\tt p117m} from problems
43, 84, 113 and 117 in\Lspace \Lcitemark 9\Rcitemark \Rspace{}
by removing certain constraints and including instead 
additional objectives of the form
$$f_i(x)=f(x)+\alpha _ig_i(x)$$
where the $\alpha _i$'s are positive scalars and $g_i(x)\leq 0.$ 
Specifically, {\tt p43m}
is constructed from problem 43 by taking out the first 
two constraints and 
including two corresponding objectives with $\alpha _i=15$ for both;
{\tt p84m} similarly corresponds to problem 84 without 
constraints 5 and 6 but
with two corresponding additional objectives, 
with $\alpha _i=20$ for both;
for {\tt p113m}, the first three linear constraints from problem 113
were turned into objectives, with $\alpha _i=10$ for all; 
for {\tt p117m},
the first two nonlinear constraints were turned into objectives, 
again with $\alpha _i=10$ for both.
The gradients of all the functions were computed by finite difference 
approximation except for {\tt polk1} through {\tt polk4} for which 
gradients were computed analytically.

In Table 2, the meaning of columns {\tt B}, {\tt nineqn}, {\tt ncallf}, 
{\tt ncallg},
{\tt iter}, {\tt ktnorm} and {\tt SCV} is as in 
Table 1 (but {\tt ncallf}
is the total number of evaluations of {\it scalar} objective function).
{\tt nf} is the number of objective functions in the max, 
{\tt objmax} is the
final value of the max of the objective functions. 
Finally, as in Table 1,
{\tt eps} is the stopping rule parameter. 
Here however its specific meaning
varies from problem to problem as we attempted to best approximate the 
stopping rule used in the reference. Specifically, 
for problems {\tt bard} through {\tt kiw4},
execution was terminated when $\|d^0_k\|$ becomes smaller 
than the corresponding value of $\epsilon$ in the column 
of {\tt eps} (this was also done for problems {\tt p43m} 
through {\tt p117m});
for problems {\tt mad1} down to {\tt mad8}, execution was
terminated when $\|d^0_k\|$ is smaller than $\|x_k\|$ times the 
corresponding value of $\epsilon$ in the column {\tt eps}
(except {\tt mad2} for which FSQPD was terminated when the 14 digits
of the maximum objective value carried out by our code did not change);
for problems {\tt polk1} through {\tt polk4}, execution 
was terminated when
$\log _e\|x_k-x^*\|$ becomes smaller than the corresponding 
value of $\epsilon$ in the column of {\tt eps}.
FSQPD with monotone line search failed to reach a solution for {\tt mad8-30}
when QLD was used, but it succeeded when QPSOL\Lspace \Lcitemark 16\Rcitemark \
Rspace{} 
was used.\footnote{But on most problems, according to our experience,
QLD is significantly faster than QPSOL. A subroutine to interface FSQP
with QPSOL can be obtained from the authors.}

Table 3 contains results of problems with nonlinear equality
constraints from\Lspace \Lcitemark 9\Rcitemark \Rspace{}. All symbols are the s
ame as
described before.  {\tt eps} is the norm requirement on $d_k^0$
and {\tt epseqn} is chosen close to the corresponding values 
in\Lspace \Lcitemark 9\Rcitemark \Rspace{}, with $10^{-8}$ replacing 0. 
An asterisk (*) indicates that FSQP failed
to meet the stopping criterion before certain execution error is encountered.
It can be checked that
the second order sufficient conditions of optimality are not satisfied
at the known optimal solution for problems 26, 27, 46 and 47.

\section{Programming Tips}
\label{tips}
\noindent
The order in which FSQP evaluates the various objectives and
constraints during the line search varies from iteration to
iteration, as the functions deemed more likely to cause rejection of
the trial steps are evaluated first.  On the other hand, in
many applications, it is far more efficient to evaluate all
(or at least more than one) of the objectives and constraints concurrently,
as they are all obtained as byproducts of expensive simulations
(e.g., involving finite element computation).  This situation
can be accomodated as follows.  Whenever a function evaluation
has been performed, store in a common block the value of {\tt x} 
and the corresponding values of all objectives and constraints (alternatively,
the values of all ``simulation outputs'').  Then, whenever a function
evaluation is requested by FSQP, first check whether the same value of 
{\tt x} has just been used and, if so, entirely bypass the expensive
simulation.   Note that, if gradients are computed by finite differences,
it will be necessary to save the past {\tt nparam}+1 values of {\tt x}
and of the corresponding objective/constraint values.

\section{Trouble-Shooting}
\label{trouble}
\noindent It is important to keep in mind some limitations of FSQP.
First, similar to most codes targeted at smooth problems, it is
likely to encounter difficulties when confronted to nonsmooth
functions such as, for example, functions involving matrix
eigenvalues. Second, because FSQP generates feasible iterates, it may
be slow if the feasible set is very ``thin'' or oddly shaped. 
Third, concerning equality constraints, if $h_j(x)\geq 0$ for all 
$x\in R^n$ and if $h_j(x_0)=0$
for some $j$ at the initial point $x_0$, the interior of the feasible set
defined by $h_j(x)\leq 0$ for such $j$ is empty. This may cause
difficulties for FSQPD because, in FSQPD, $h_j(x)=0$ is directly 
turned into $h_j(x)\leq 0$ for such $j$.
The user is advised to either give an initial point
that is infeasible for all nonlinear equality constraints or change
the sign of $h_j$ so that $h_j(x)<0$ can be achieved at some point
for all such nonlinear equality constraint.

A common failure mode for FSQP, corresponding to ${\tt inform}=5$ or 6,
is that of the QP solver in constructing {\tt d0} or {\tt d1}.
This is often due to linear dependence (or almost dependence) 
of gradients of equality constraints or active inequality constraints.
Sometimes this problem can be circumvented by making use of a more
robust (but likely slower) QP solver.  We have designed an interface,
available upon request, that allows the user to use QPSOL\Lspace \Lcitemark 16\
Rcitemark \Rspace{}
instead of QLD. The user may also want to
check the jacobian matrix and identify which constraints are the
culprit.  Eliminating redundant constraints or formulating the constraints 
differently (without changing the feasible set) may then be the way to go.

Finally, when FSQP fails in the line search ({\tt inform}=4), it is
typically due to inaccurate computation of the search direction.  Two
possible reasons are: (i) Insufficient accuracy of the QP solver; again,
it may be appropriate to substitute a different QP solver.  (ii)
Insufficient accuracy of gradient computation, e.g., when gradients
are computed by finite differences.  A remedy may be to provide
analytical gradients or, more astutely, to resort to ``automatic
differentiation''.

\vspace{1em}
\noindent{\bf Acknowledgment}

\vspace{1em}
The authors are indebted to Dr. E.R. Panier for many invaluable
comments and suggestions.

\vspace{1em}
\noindent{\bf References}
\vspace{1em}

\message{REFERENCE LIST}

\bgroup\Resetstrings%
\def\Loccittest{}\def\Abbtest{}\def\Capssmallcapstest{}\def\Edabbtest{}\def\Edc
apsmallcapstest{}\def\Underlinetest{}%
\def\NoArev{0}\def\NoErev{0}\def\Acnt{2}\def\Ecnt{0}\def\acnt{0}\def\ecnt{0}%
\def\Ftest{ }\def\Fstr{1}%
\def\Atest{ }\def\Astr{D.Q. Mayne%
  \Aand E. Polak}%
\def\Ttest{ }\def\Tstr{Feasible Directions Algorithms for Optimization Problems
 with Equality and Inequality Constraints}%
\def\Jtest{ }\def\Jstr{Math. Programming}%
\def\Vtest{ }\def\Vstr{11}%
\def\Ptest{ }\def\Pcnt{ }\def\Pstr{67--80}%
\def\Dtest{ }\def\Dstr{1976}%
\Refformat\egroup%

\bgroup\Resetstrings%
\def\Loccittest{}\def\Abbtest{}\def\Capssmallcapstest{}\def\Edabbtest{}\def\Edc
apsmallcapstest{}\def\Underlinetest{}%
\def\NoArev{0}\def\NoErev{0}\def\Acnt{2}\def\Ecnt{0}\def\acnt{0}\def\ecnt{0}%
\def\Ftest{ }\def\Fstr{2}%
\def\Atest{ }\def\Astr{E.R. Panier%
  \Aand A.L. Tits}%
\def\Ttest{ }\def\Tstr{On Combining Feasibility, Descent and Superlinear Conver
gence in Inequality Constrained Optimization}%
\def\Jtest{ }\def\Jstr{Math. Programming}%
\def\Dtest{ }\def\Dstr{1993, to appear}%
\Refformat\egroup%

\bgroup\Resetstrings%
\def\Loccittest{}\def\Abbtest{}\def\Capssmallcapstest{}\def\Edabbtest{}\def\Edc
apsmallcapstest{}\def\Underlinetest{}%
\def\NoArev{0}\def\NoErev{0}\def\Acnt{4}\def\Ecnt{0}\def\acnt{0}\def\ecnt{0}%
\def\Ftest{ }\def\Fstr{3}%
\def\Atest{ }\def\Astr{J.F. Bonnans%
  \Acomma E.R. Panier%
  \Acomma A.L. Tits%
  \Aandd J.L. Zhou}%
\def\Ttest{ }\def\Tstr{Avoiding the Maratos Effect by Means of a Nonmonotone Li
ne Search. II. Inequality Constrained Problems -- Feasible Iterates}%
\def\Jtest{ }\def\Jstr{SIAM J. Numer. Anal.}%
\def\Vtest{ }\def\Vstr{29}%
\def\Ntest{ }\def\Nstr{4}%
\def\Dtest{ }\def\Dstr{1992}%
\def\Ptest{ }\def\Pcnt{ }\def\Pstr{1187--1202}%
\Refformat\egroup%

\bgroup\Resetstrings%
\def\Loccittest{}\def\Abbtest{}\def\Capssmallcapstest{}\def\Edabbtest{}\def\Edc
apsmallcapstest{}\def\Underlinetest{}%
\def\NoArev{0}\def\NoErev{0}\def\Acnt{2}\def\Ecnt{0}\def\acnt{0}\def\ecnt{0}%
\def\Ftest{ }\def\Fstr{4}%
\def\Atest{ }\def\Astr{J.L. Zhou%
  \Aand A.L. Tits}%
\def\Ttest{ }\def\Tstr{Nonmonotone Line Search for Minimax Problems}%
\def\Jtest{ }\def\Jstr{J. Optim. Theory Appl.}%
\def\Vtest{ }\def\Vstr{76}%
\def\Ntest{ }\def\Nstr{3}%
\def\Ptest{ }\def\Pcnt{ }\def\Pstr{455--476}%
\def\Dtest{ }\def\Dstr{1993}%
\Refformat\egroup%

\bgroup\Resetstrings%
\def\Loccittest{}\def\Abbtest{}\def\Capssmallcapstest{}\def\Edabbtest{}\def\Edc
apsmallcapstest{}\def\Underlinetest{}%
\def\NoArev{0}\def\NoErev{0}\def\Acnt{3}\def\Ecnt{0}\def\acnt{0}\def\ecnt{0}%
\def\Ftest{ }\def\Fstr{5}%
\def\Atest{ }\def\Astr{L. Grippo%
  \Acomma F. Lampariello%
  \Aandd S. Lucidi}%
\def\Ttest{ }\def\Tstr{A Nonmonotone Line Search Technique for Newton's Method}
%
\def\Jtest{ }\def\Jstr{SIAM J. Numer. Anal.}%
\def\Vtest{ }\def\Vstr{23}%
\def\Ntest{ }\def\Nstr{4}%
\def\Ptest{ }\def\Pcnt{ }\def\Pstr{707--716}%
\def\Dtest{ }\def\Dstr{1986}%
\Refformat\egroup%

\bgroup\Resetstrings%
\def\Loccittest{}\def\Abbtest{}\def\Capssmallcapstest{}\def\Edabbtest{}\def\Edc
apsmallcapstest{}\def\Underlinetest{}%
\def\NoArev{0}\def\NoErev{0}\def\Acnt{2}\def\Ecnt{0}\def\acnt{0}\def\ecnt{0}%
\def\Ftest{ }\def\Fstr{6}%
\def\Atest{ }\def\Astr{D. Q. Mayne%
  \Aand E. Polak}%
\def\Ttest{ }\def\Tstr{A Superlinearly Convergent Algorithm for Constrained Opt
imization Problems}%
\def\Jtest{ }\def\Jstr{Math. Programming Stud.}%
\def\Vtest{ }\def\Vstr{16}%
\def\Ptest{ }\def\Pcnt{ }\def\Pstr{45--61}%
\def\Dtest{ }\def\Dstr{1982}%
\Refformat\egroup%

\bgroup\Resetstrings%
\def\Loccittest{}\def\Abbtest{}\def\Capssmallcapstest{}\def\Edabbtest{}\def\Edc
apsmallcapstest{}\def\Underlinetest{}%
\def\NoArev{0}\def\NoErev{0}\def\Acnt{1}\def\Ecnt{0}\def\acnt{0}\def\ecnt{0}%
\def\Ftest{ }\def\Fstr{7}%
\def\Atest{ }\def\Astr{K. Schittkowski}%
\def\Ttest{ }\def\Tstr{QLD : A FORTRAN Code for Quadratic Programming, User's G
uide}%
\def\Itest{ }\def\Istr{Mathematisches Institut, Universit{\"a}t Bayreuth}%
\def\Ctest{ }\def\Cstr{Germany}%
\def\Dtest{ }\def\Dstr{1986}%
\Refformat\egroup%

\bgroup\Resetstrings%
\def\Loccittest{}\def\Abbtest{}\def\Capssmallcapstest{}\def\Edabbtest{}\def\Edc
apsmallcapstest{}\def\Underlinetest{}%
\def\NoArev{0}\def\NoErev{0}\def\Acnt{1}\def\Ecnt{1}\def\acnt{0}\def\ecnt{0}%
\def\Ftest{ }\def\Fstr{8}%
\def\Atest{ }\def\Astr{M.J.D. Powell}%
\def\Ttest{ }\def\Tstr{A Fast Algorithm for Nonlinearly Constrained Optimizatio
n Calculations}%
\def\Btest{ }\def\Bstr{Numerical Analysis, Dundee, 1977, Lecture Notes in Mathe
matics 630}%
\def\Etest{ }\def\Estr{G.A. Watson}%
\def\Itest{ }\def\Istr{Springer-Verlag}%
\def\Ptest{ }\def\Pcnt{ }\def\Pstr{144--157}%
\def\Dtest{ }\def\Dstr{1978}%
\Refformat\egroup%

\bgroup\Resetstrings%
\def\Loccittest{}\def\Abbtest{}\def\Capssmallcapstest{}\def\Edabbtest{}\def\Edc
apsmallcapstest{}\def\Underlinetest{}%
\def\NoArev{0}\def\NoErev{0}\def\Acnt{2}\def\Ecnt{0}\def\acnt{0}\def\ecnt{0}%
\def\Ftest{ }\def\Fstr{9}%
\def\Atest{ }\def\Astr{W. Hock%
  \Aand K. Schittkowski}%
\def\Ttest{ }\def\Tstr{Test Examples for Nonlinear Programming Codes}%
\def\Itest{ }\def\Istr{Lecture Notes in Economics and Mathematical Systems (187
), Springer Verlag}%
\def\Dtest{ }\def\Dstr{1981}%
\Refformat\egroup%

\bgroup\Resetstrings%
\def\Loccittest{}\def\Abbtest{}\def\Capssmallcapstest{}\def\Edabbtest{}\def\Edc
apsmallcapstest{}\def\Underlinetest{}%
\def\NoArev{0}\def\NoErev{0}\def\Acnt{2}\def\Ecnt{0}\def\acnt{0}\def\ecnt{0}%
\def\Ftest{ }\def\Fstr{10}%
\def\Atest{ }\def\Astr{K. Madsen%
  \Aand H. Schj{\ae}r-Jacobsen}%
\def\Ttest{ }\def\Tstr{Linearly Constrained Minimax Optimization}%
\def\Jtest{ }\def\Jstr{Math. Programming}%
\def\Vtest{ }\def\Vstr{14}%
\def\Ptest{ }\def\Pcnt{ }\def\Pstr{208--223}%
\def\Dtest{ }\def\Dstr{1978}%
\Refformat\egroup%

\bgroup\Resetstrings%
\def\Loccittest{}\def\Abbtest{}\def\Capssmallcapstest{}\def\Edabbtest{}\def\Edc
apsmallcapstest{}\def\Underlinetest{}%
\def\NoArev{0}\def\NoErev{0}\def\Acnt{1}\def\Ecnt{0}\def\acnt{0}\def\ecnt{0}%
\def\Ftest{ }\def\Fstr{11}%
\def\Atest{ }\def\Astr{G.A. Watson}%
\def\Ttest{ }\def\Tstr{The Minimax Solution of an Overdetermined System of Non-
linear Equations}%
\def\Jtest{ }\def\Jstr{J. Inst. Math. Appl.}%
\def\Vtest{ }\def\Vstr{23}%
\def\Ptest{ }\def\Pcnt{ }\def\Pstr{167--180}%
\def\Dtest{ }\def\Dstr{1979}%
\Refformat\egroup%

\bgroup\Resetstrings%
\def\Loccittest{}\def\Abbtest{}\def\Capssmallcapstest{}\def\Edabbtest{}\def\Edc
apsmallcapstest{}\def\Underlinetest{}%
\def\NoArev{0}\def\NoErev{0}\def\Acnt{2}\def\Ecnt{0}\def\acnt{0}\def\ecnt{0}%
\def\Ftest{ }\def\Fstr{12}%
\def\Atest{ }\def\Astr{C. Charalambous%
  \Aand A.R. Conn}%
\def\Ttest{ }\def\Tstr{An Efficient Method to Solve the Minimax Problem Directl
y}%
\def\Jtest{ }\def\Jstr{SIAM J. Numer. Anal.}%
\def\Vtest{ }\def\Vstr{15}%
\def\Ntest{ }\def\Nstr{1}%
\def\Ptest{ }\def\Pcnt{ }\def\Pstr{162--187}%
\def\Dtest{ }\def\Dstr{1978}%
\Refformat\egroup%

\bgroup\Resetstrings%
\def\Loccittest{}\def\Abbtest{}\def\Capssmallcapstest{}\def\Edabbtest{}\def\Edc
apsmallcapstest{}\def\Underlinetest{}%
\def\NoArev{0}\def\NoErev{0}\def\Acnt{2}\def\Ecnt{0}\def\acnt{0}\def\ecnt{0}%
\def\Ftest{ }\def\Fstr{13}%
\def\Atest{ }\def\Astr{A.R. Conn%
  \Aand Y. Li}%
\def\Ttest{ }\def\Tstr{An Efficient Algorithm for Nonlinear Minimax Problems}%
\def\Rtest{ }\def\Rstr{Research Report CS-88-41}%
\def\Itest{ }\def\Istr{University of Waterloo}%
\def\Ctest{ }\def\Cstr{Waterloo, Ontario, N2L 3G1 Canada}%
\def\Dtest{ }\def\Dstr{November, 1989 }%
\Refformat\egroup%

\bgroup\Resetstrings%
\def\Loccittest{}\def\Abbtest{}\def\Capssmallcapstest{}\def\Edabbtest{}\def\Edc
apsmallcapstest{}\def\Underlinetest{}%
\def\NoArev{0}\def\NoErev{0}\def\Acnt{1}\def\Ecnt{0}\def\acnt{0}\def\ecnt{0}%
\def\Ftest{ }\def\Fstr{14}%
\def\Atest{ }\def\Astr{K.C. Kiwiel}%
\def\Ttest{ }\def\Tstr{Methods of Descent in Nondifferentiable Optimization}%
\def\Stest{ }\def\Sstr{Lecture Notes in Mathematics}%
\def\Ntest{ }\def\Nstr{1133}%
\def\Itest{ }\def\Istr{Springer-Verlag}%
\def\Ctest{ }\def\Cstr{New York--Heidelberg--Berlin}%
\def\Ctest{ }\def\Cstr{Berlin, Heidelberg, New-York, Tokyo}%
\def\Dtest{ }\def\Dstr{1985}%
\Refformat\egroup%

\bgroup\Resetstrings%
\def\Loccittest{}\def\Abbtest{}\def\Capssmallcapstest{}\def\Edabbtest{}\def\Edc
apsmallcapstest{}\def\Underlinetest{}%
\def\NoArev{0}\def\NoErev{0}\def\Acnt{3}\def\Ecnt{0}\def\acnt{0}\def\ecnt{0}%
\def\Ftest{ }\def\Fstr{15}%
\def\Atest{ }\def\Astr{E. Polak%
  \Acomma D.Q. Mayne%
  \Aandd J.E. Higgins}%
\def\Ttest{ }\def\Tstr{A Superlinearly Convergent Algorithm for Min-max Problem
s}%
\def\Jtest{ }\def\Jstr{Proceedings of the 28th IEEE Conference on Decision and 
Control}%
\def\Dtest{ }\def\Dstr{December 1989}%
\def\Ptest{ }\def\Pcnt{ }\def\Pstr{894--898}%
\Refformat\egroup%

\bgroup\Resetstrings%
\def\Loccittest{}\def\Abbtest{}\def\Capssmallcapstest{}\def\Edabbtest{}\def\Edc
apsmallcapstest{}\def\Underlinetest{}%
\def\NoArev{0}\def\NoErev{0}\def\Acnt{4}\def\Ecnt{0}\def\acnt{0}\def\ecnt{0}%
\def\Ftest{ }\def\Fstr{16}%
\def\Atest{ }\def\Astr{P.E. Gill%
  \Acomma W. Murray%
  \Acomma M.A. Saunders%
  \Aandd M.H. Wright}%
\def\Ttest{ }\def\Tstr{User's Guide for QPSOL (Version 3.2): A Fortran Package 
for Quadratic Programming}%
\def\Rtest{ }\def\Rstr{Technical Report SOL 84-6}%
\def\Itest{ }\def\Istr{Systems Optimization Laboratory, Stanford University}%
\def\Ctest{ }\def\Cstr{Stanford, CA 94305}%
\def\Dtest{ }\def\Dstr{1984}%
\Refformat\egroup%


\newpage
\renewcommand{\baselinestretch}{0.95} % more interline spacing
 \footnotesize{
\begin{tabular}{ccccccrlll} \hline
\multicolumn{1}{c}{{\tt prob}} & 
\multicolumn{1}{c}{{\tt B}} & 
\multicolumn{1}{c}{{\tt nineqn}} & 
\multicolumn{1}{c}{{\tt ncallf}} & 
\multicolumn{1}{c}{{\tt ncallg}} & 
\multicolumn{1}{c}{{\tt iter}} &
\multicolumn{1}{c}{{\tt objective}} & 
\multicolumn{1}{c}{{\tt ktnorm}} & 
\multicolumn{1}{c}{{\tt eps}} & 
\multicolumn{1}{l}{{\tt SCV}} \\ \hline \\

 {\tt p12} & 0 & 1  & ~7   &  ~14  &~7  &$-$.300000000E+02 &.72E-06 & .10E-05 &
.0\\
           & 1 &    & ~7   &  ~12  &~7  &$-$.300000000E+02 &.79E-06 & .10E-05 &
.0\\\hline
 {\tt p29} & 0 & 1  &  11  &  ~20  & 10 &$-$.226274170E+02 &.41E-05 & .10E-04 &
.0\\
           & 1 &    &  12  &  ~16  & 12 &$-$.226274170E+02 &.63E-05 & .10E-04 &
.0\\\hline
 {\tt p30} & 0 & 1  &  13  &  ~25  & 13 &   .100000000E+01 &.26E-07 & .10E-06 &
.0\\
           & 1 &    &  14  &  ~14  & 14 &   .100000000E+01 &.43E-07 & .10E-06 &
.0\\\hline
 {\tt p31} & 0 & 1  &  10  &  ~21  &~8  &   .600000000E+01 &.34E-06 & .10E-04 &
.0\\
           & 1 &    &  10  &  ~18  & 10 &   .600000000E+01 &.50E-06 & .10E-04 &
.0\\\hline
 {\tt p32} & 0 & 1  & ~3   & ~~5   &~3  &   .100000000E+01 &.15E-14 & .10E-07 &
.0\\
           & 1 &    & ~3   & ~~4   &~3  &   .100000000E+01 &.64E-16 & .10E-07 &
.0\\\hline
 {\tt p33} & 0 & 2  & ~4   &  ~11  &~4  &$-$.400000000E+01 &.13E-11 & .10E-07 &
.0\\
           & 1 &    & ~5   &  ~10  &~5  &$-$.400000000E+01 &.47E-11 & .10E-07 &
.0\\\hline
 {\tt p34} & 0 & 2  & ~7   &  ~28  &~7  &$-$.834032443E+00 &.19E-08 & .10E-07 &
.0\\
           & 1 &    & ~9   &  ~24  &~9  &$-$.834032445E+00 &.38E-11 & .10E-07 &
.0\\\hline
 {\tt p43} & 0 & 3  &  11  &  ~51  & ~9 &$-$.440000000E+02 &.12E-05 & .10E-04 &
.0\\
           & 1 &    &  12  &  ~49  &12  &$-$.440000000E+02 &.16E-06 & .10E-04 &
.0\\\hline
 {\tt p44} & 0 & 0  &  ~6  &  ~~0  & ~6 &$-$.150000000E+02 &.0      & .10E-07 &
.0\\
           & 1 &    &  ~6  &       &~6  &$-$.150000000E+02 &.0      & .10E-07 &
.0\\\hline
 {\tt p51} & 0 & 0  &  ~8  & ~~0   &~6  &   .505655658E$-$15 &.46E-06 &.10E-05 
&.22E-15\\
           & 1 &    &  ~9  &       &~8  &   .505655658E$-$15 &.34E-08 &.10E-05 
&.22E-15\\\hline
 {\tt p57} & 0 & 1  &  ~5  &  ~~7  &~3  &   .306463061E$-$01 &.29E-05 &.10E-04 
&.0\\
           & 1 &    &  ~5  &  ~~7  &~3  &   .306463061E$-$01 &.28E-05 &.10E-04 
&.0\\\hline
 {\tt p66} & 0 & 2  & ~8   &  ~30  &~8  &   .518163274E+00   &.50E-09 &.10E-07 
&.0\\
           & 1 &    & ~9   &  ~24  &~9  &   .518163274E+00   &.14E-08 &.10E-07 
&.0\\\hline
 {\tt p67} & 0 & 14 & 21   &  305  &21  &$-$.116211927E+02   &.88E-06 &.10E-04 
&.0\\
           & 1 &    & 61   &  854  &61  &$-$.116211927E+02   &.58E-05 &.10E-04 
&.0\\\hline
 {\tt p70} & 0 & 1  & 32   &  ~39  &30  &   .940197325E$-$02 &.58E-08 &.10E-06 
&.0\\
           & 1 &    & 31   &  ~31  &31  &   .940197325E$-$02 &.19E-07 &.10E-06 
&.0\\\hline
 {\tt p76} & 0 & 0  & ~6   & ~~0   &~6  &$-$.468181818E+01 &.34E-04 &.10E-03 &.
0\\
           & 1 &    & ~6   &       &~6  &$-$.468181818E+01 &.34E-04 &.10E-03 &.
0\\\hline
 {\tt p84} & 0 & 6  & ~4   &  ~30  &~4  &$-$.528033513E+07 &.0      & .10E-07 &
.0\\
*          & 1 &    & ~4   &  ~29  &~4  &$-$.528033513E+07 &.38E-09 & .10E-07 &
.0\\\hline
 {\tt p85} & 0 &38  & 34   & 1347  &34  &$-$.240000854E+01 &.35E-03 &.10E-02 &.
0\\
           & 1 &    & 80   & 3040  &80  &$-$.240000854E+01 &.81E-03 &.10E-02 &.
0\\\hline
 {\tt p86} & 0 & 0  & ~8   & ~~0   &~6  &$-$.323486790E+02 &.22E-08 & .10E-05 &
.0\\
           & 1 &    & ~7   &       &~6  &$-$.323486790E+02 &.53E-06 & .10E-05 &
.0\\\hline
 {\tt p93} & 0 & 2  &  15  &  ~58  & 12 &   .135075968E+03 &.37E-03 & .10E-02 &
.0\\
           & 1 &    &  15  &  ~36  & 15 &   .135075964E+03 &.24E-04 & .10E-02 &
.0\\\hline
 {\tt p100}& 0 & 4  &  23  &   114 & 16 &   .680630057E+03 &.62E-06 & .10E-03 &
.0\\
           & 1 &    &  20  &   102 & 17 &   .680630057E+03 &.49E-04 & .10E-03 &
.0\\\hline
 {\tt p110}& 0 & 0  &  ~9  & ~~0   &~8  &$-$.457784697E+02 &.50E-06 & .10E-05 &
.0\\
           & 1 &    &  ~9  &       &~8  &$-$.457784697E+02 &.50E-06 & .10E-05 &
.0\\\hline
 {\tt p113}& 0 & 5  &  12  &  108  & 12 &   .243063768E+02 &.81E-03 & .10E-02 &
.0\\
           & 1 &    &  12  &  ~99  & 12 &   .243064357E+02 &.83E-03 & .10E-02 &
.35E-14\\\hline
 {\tt p117}& 0 & 5  &  20  &  219  & 19 &   .323486790E+02 &.58E-04 & .10E-03 &
.0\\
           & 1 &    &  18  &  ~93  & 17 &   .323486790E+02 &.34E-04 & .10E-03 &
.0\\\hline
 {\tt p118}& 0 & 0  &  19  &~~0    & 19 &   .664820450E+03 &.13E-14 & .10E-07 &
.0\\ 
           & 1 &    &  19  &       & 19 &   .664820450E+03 &.17E-14 & .10E-07 &
.0\\ 
\hline
\end{tabular}
}

\nopagebreak
\vspace{1em}
\hspace{4em}
Table 1: Results for Inequality Constrained Problems with FSQP Version 3.2

\newpage
\renewcommand{\baselinestretch}{1.0} % more interline spacing
 \begin{quote}
{\scriptsize
\begin{tabular}{cccccccrllc} \hline
\multicolumn{1}{c}{\tt prob} & 
\multicolumn{1}{c}{{\tt B}} & 
\multicolumn{1}{c}{{\tt nineqn}} & 
\multicolumn{1}{r}{{\tt nf}} & 
\multicolumn{1}{c}{{\tt ncallf}} & 
\multicolumn{1}{c}{{\tt ncallg}} &
\multicolumn{1}{c}{{\tt iter}} &
\multicolumn{1}{c}{{\tt objmax}} & 
\multicolumn{1}{c}{{\tt ktnorm}} &
\multicolumn{1}{c}{{\tt eps}} &
\multicolumn{1}{c}{{\tt SCV}}\\ \hline \\
 {\tt bard} & 0 &0&~15&~168&~~0&~8&   .508163265E$-$01&.61E-09 & .50E-05&.0 \\
            & 1 & &   &~105&   &~7&   .508168686E$-$01&.22E-06 & .50E-05&.0 \\\
hline
 {\tt cb2}  & 0 &0&~~3&~~30&~~0&~6&   .195222449E+01  &.37E-06 & .50E-05&.0 \\
            & 1 & &   &~~18&   &~6&   .195222449E+01  &.29E-05 & .50E-05 &.0\\\
hline
 {\tt cb3}  & 0 &0&~~3&~~15&~~0&~3&   .200000157E+01  &.40E-05 & .50E-05 &.0\\
            & 1 & &   &~~15&   &~5&   .200000000E+01  &.47E-08 & .50E-05 &.0\\\
hline
{\tt colv}  & 0 &0&~~6&~240&~~0&21&   .323486790E+02  &.46E-05 & .50E-05 &.0\\
            & 1 & &   &~102&   &17&   .323486790E+02  &.12E-04 & .50E-05 &.0\\\
hline
{\tt davd2} & 0 &0&~20&~342&~~0&12&   .115706440E+03  &.62E-06 & .50E-05 &.0\\
            & 1 & &   &~220&   &11&   .115706440E+03  &.11E-05 & .50E-05 &.0\\\
hline
{\tt f\&r}  & 0 &0&~~2&~~32&~~0&~9&   .494895210E+01  &.90E-09 & .50E-05 &.0\\
            & 1 & &   &~~20&   &10&   .494895210E+01  &.70E-07 & .50E-05 &.0\\\
hline
{\tt hettich}& 0 &0&~~5&~125&~~0&13&  .245935695E$-$02&.10E-07 & .50E-05&.0 \\
            & 1 & &   &~~75&   &11&   .245936698E$-$02&.18E-07 & .50E-05&.0 \\\
hline
{\tt r-s}   & 0 &0&~~4&~~71&~~0&~9&$-$.440000000E+02  &.98E-06 & .50E-05 &.0\\
            & 1 & &   &~~68&   &12&$-$.440000000E+02  &.28E-06 & .50E-05 &.0\\\
hline
{\tt wats-6}& 0 &0&~31&~623&~~0&12&   .127172748E$-$01&.42E-07 & .50E-05&.0 \\
            & 1 & &   &~433&   &13&   .127170913E$-$01&.84E-10 & .50E-05 &.0\\\
hline
{\tt wats-20}& 0 &0&~31&1953&~~0&32&  .895554035E$-$07&.13E-05 & .50E-05 &.0\\
            & 1 & &   &1023&   &32&   .898278737E$-$07&.13E-05 & .50E-05 &.0\\\
hline
{\tt wong}  & 0 &0&~~5&~182&~~0&19&   .680630057E+03  &.40E-04 & .50E-05 &.0\\
            & 1 & &   &~171&   &26&   .680630057E+03  &.13E-03 & .50E-05 &.0\\\
hline
{\tt kiw1}  & 0 &0&~10&~159&~~0&11&   .226001621E+02  &.32E-05 & .11E-05 &.0\\
            & 1 & &   &~130&   &13&   .226001621E+02  &.54E-05 & .60E-06 &.0\\\
hline
{\tt kiw4}  & 0 &0&~~2&~~42&~~0&~9&   .222044605E$-$15&.18E-07 &.42E-07&.0  \\
            & 1 & &   &~~23&   &~9&  .0\hspace{5.75em}~~~&.47E-07 &.15E-07 &.0 
\\\hline
{\tt mad1}  & 0 &0&~~3&~~24&~~0&~5&$-$.389659516E+00  &.22E-10 & .10E-09 &.0\\
            & 1 & &   &~~18&   &~6&$-$.389659516E+00  &.48E-10 & .10E-09 &.0\\\
hline
{\tt mad2}  & 0 &0&~~3&~~25&~~0&~5&$-$.330357143E+00  &.22E-10 & .10E-09 &.0\\
            & 1 & &   &~~21&   &~6&$-$.330357143E+00  &.86E-09 & .10E-09 &.0\\\
hline
{\tt mad4}  & 0 &0&~~3&~~29&~~0&~6&$-$.448910786E+00  &.31E-17 & .10E-09 &.0\\
            & 1 & &   &~~24&   &~8&$-$.448910786E+00  &.38E-16 & .10E-09 &.0\\\
hline
{\tt mad5}  & 0 &0&~~3&~~31&~~0&~7&$-$.100000000E+01  &.21E-11 & .10E-09 &.0\\
            & 1 & &   &~~24&   &~8&$-$.100000000E+01  &.78E-14 & .10E-09 &.0\\\
hline
{\tt mad6}  & 0 &0&163&1084&~~0&~6&   .113104727E+00  &.81E-11 & .10E-09 &.0\\
            & 1 & &   &1141&   &~7&   .113104727E+00  &.21E-10 & .10E-09 &.0\\\
hline
{\tt mad8-10}& 0&0&~18&~291&~~0&10&   .381173963E+00  &.89E-11 & .10E-09 &.0\\
            & 1 & &   &~252&   &14&   .381173963E+00  &.16E-14 & .10E-09 &.0\\\
hline
{\tt mad8-30}& 0&0&   &    &   & *&                   &        & .10E-09 &  \\
            & 1 & &   &1102&   &18&   .547620496E+00  &.12E-14 & .10E-09 &.0\\\
hline
{\tt mad8-50}& 0&0&~98&3056&~~0&21&   .579276202E+00  &.86E-15 & .10E-09 &.0\\
            & 1 & &   &2084&   &21&   .579276202E+00  &.91E-16 & .10E-09 &.0\\\
hline
{\tt polk1} & 0 &0&~~2&~~42&~~0&10&   .271828183E+01  &.50E-04 & ~$-$10.00&.0 \
\
            & 1 & &   &~~22&   &10&   .271828183E+01  &.68E-04 & ~$-$10.00&.0 \
\\hline
{\tt polk2} & 0 &0&~~2&~203&~~0&42&   .545981839E+02  &.28E-03 & ~$-$\,~9.00&.0
 \\
            & 1 & &   &~116&   &38&   .545981500E+02  &.14E-02 & ~$-$\,~9.00&.0
 \\\hline
{\tt polk3} & 0 &0&~10&~188&~~0&12&   .370348302E+01  &.23E-02 & ~$-$\,~5.50&.0
 \\
            & 1 & &   &~141&   &12&   .370348272E+01  &.26E-02 & ~$-$\,~5.50&.0
 \\\hline
{\tt polk4} & 0 &0&~~3&~~45&~~0&~7&.0\hspace{5.45em}~~~  &.39E-04 & ~$-$10.00&.
0 \\
            & 1 & &   &~~24&   &~7&   .364604254E+00  &.37E-06 & ~$-$10.00&.0 \
\\hline
{\tt p43m}  & 0 &1&~~3&~~80&~43&15&$-$.440000000E+02  &.14E-05 & .50E-05&.0\\
            &1  & &   &~~63&~25&16&$-$.440000000E+02  &.46E-05 & .50E-05&.0\\\h
line
{\tt p84m}  &0  &4&~~3&~~17&~20&~4&$-$.528033513E+07  &.28E-09 & .50E-05&.0\\
            &1  & &   &~~~9&~12&~3&$-$.528033511E+07  &.76E-05 & .50E-05&.0\\\h
line
{\tt p113m} &0  &5&~~4&~108&127&14&   .243062091E+02  &.14E-04 & .50E-05&.0\\
            &1  & &   &~~84&105&14&   .243062091E+02  &.29E-04 & .50E-05&.0\\\h
line
{\tt p117m} &0  &3&~~3&~124&144&21&   .323486790E+02  &.43E-05 & .50E-05&.0\\
            &1  & &   &~~57&~54&17&   .323486790E+02  &.26E-04 & .50E-05&.0\\\h
line
\end{tabular}
}
\end{quote}

\nopagebreak
\hspace{6em}Table 2: Results for Minimax Problems with FSQP Version 3.2

\newpage
\renewcommand{\baselinestretch}{0.924} % more interline spacing
 \footnotesize{
\begin{tabular}{cccccrllll} \hline
\multicolumn{1}{c}{{\tt prob}} & 
\multicolumn{1}{c}{{\tt B}} & 
\multicolumn{1}{c}{{\tt ncallf}} & 
\multicolumn{1}{c}{{\tt ncallg}} & 
\multicolumn{1}{c}{{\tt iter}} &
\multicolumn{1}{c}{{\tt objective}} & 
\multicolumn{1}{c}{{\tt ktnorm}} & 
\multicolumn{1}{c}{{\tt eps}} & 
\multicolumn{1}{c}{{\tt epseqn}} & 
\multicolumn{1}{c}{{\tt SCV}} \\ \hline \\

 {\tt  p6} &0 &~17   &  ~22  &10  &   .274055126E$-$11 &.42E-05 & .10E-03 &.40E
-06&.20E-09\\
           &1 &~21   &  ~23  &10  &   .116074629E$-$12 &.35E-05 & .10E-03 &.40E
-06&.28E-06\\\hline
 {\tt  p7} &0 &~57   &  ~57  &13  &$-$.173205081E+01 &.12E-06 & .10E-03 &.35E-0
8&.70E-09\\
           &1 & ~27  &  ~25  & 15 &$-$.173205081E+01 &.68E-08 & .10E-03 &.35E-0
8&.15E-09\\\hline
 {\tt p26} &0 & 127  &  138  & 51 &   .270576724E$-$13 &.15E-08 & .10E-03 &.16E
-04&.12E-09\\
           &1 & ~38  &  ~38  & 31 &   .322181110E$-$13 &.49E-08 & .10E-03 &.16E
-04&.43E-08\\\hline
 {\tt p27} &0 & 153 &  147  & 44 &   .399986835E$-$01 &.24E-02 & .10E-02 &.10E-
02&.38E-04\\
           &1 & 999  &  996  &130 &   .399916645E$-$01 &.39E-03 & .10E-02 &.10E
-02&.21E-03\\\hline
 {\tt p39} &0 &~23   & ~49   & 17 &$-$.100000000E+01 &.39E-04 & .10E-03 &.75E-0
4&.90E-08\\
           &1 &~12   & ~25   &12  &$-$.100000000E+01 &.50E-04 & .10E-03 &.75E-0
4&.64E-06\\\hline
 {\tt p40} &0 &~~5   &  ~15  & ~5 &$-$.250000002E+01 &.26E-05 & .10E-03 &.85E-0
4&.96E-08\\
           &1 &~~5   &  ~17  &~5  &$-$.250000000E+01 &.41E-04 & .10E-03 &.85E-0
4&.43E-05\\\hline
 {\tt p42} &0 &~~9   &  ~10  &~6  &   .138578644E+02 &.27E-05 & .10E-03 &.45E-0
5&.51E-09\\
           &1 &~~7   &  ~12  &~7  &   .138578652E+02 &.26E-03 & .10E-03 &.45E-0
5&.33E-06\\\hline
 {\tt p46} &0 & ~62  &  135  &26  &   .224262538E$-$10 &.11E-04 & .10E-03 &.50E
-04&.57E-10  \\
           &1 & ~56  &  ~25  &14  &   .461984187E$-$04 &.19E-02 & .10E-03 &.50E
-04&.95E-06\\\hline
 {\tt p47} &0 & ~74  &  241  &38  &   .162241544E$-$11 &.56E-06 & .10E-03 &.60E
-04&.41E-09\\
           &1 & ~50  &  282  & 36 &   .308185534E$-$01 &.11E-04 & .10E-03 &.60E
-04&.26E-08  \\\hline
 {\tt p56} &0 & ~31  & 147   &15  &$-$.345600000E+01 &.46E-08 & .10E-03 &.25E-0
6&.34E-10\\
           &1 & ~14  & ~60   &14  &$-$.345600000E+01 &.88E-05 & .10E-03 &.25E-0
6&.11E-08\\\hline
 {\tt p60} &0 & ~10  &  ~13  &10  &   .325682003E$-$01 &.29E-05 &.10E-03 &.55E-
04&.27E-09\\
           &1 & ~~9  &  ~14  &~9  &   .325687946E$-$01 &.21E-03 &.10E-03 &.55E-
04&.55E-04\\\hline
 {\tt p61} &0 &~18   &  ~38  &~8  &$-$.143646142E+03   &.35E-04 &.10E-03 &.25E-
06&.13E-07\\
           &1 &~38   &  ~17  &~9  &$-$.143646142E+03   &.67E-07 &.10E-03 &.25E-
06&.27E-12\\\hline
 {\tt p63} &0 &~~8   &  ~10  &~8  &   .961715172E+03   &.12E-06 &.10E-03 &.60E-
05&.15E-10\\
           &1 &~~6   &  ~10  &~6  &   .961715172E+03   &.25E-04 &.10E-03 &.60E-
05&.65E-07\\\hline
 {\tt p71} &0 &~~9   &  ~24  &~8  &   .170140173E+02   &.34E-07 &.10E-03 &.70E-
05&.35E-11\\
           &1 &~~6   &  ~19  &~6  &   .170140173E+02   &.79E-09 &.10E-03 &.70E-
05&.28E-08\\\hline
 {\tt p74} &0 &~14   & ~43   &14  &   .512649811E+04 &.65E-06 &.10E-03 &.65E-05
&.21E-10\\
           &1 &~41   & 123   &41  &   .512649811E+04 &.31E-04 &.10E-03 &.65E-05
&.16E-08\\\hline
 {\tt p75} &0 &~13   &  ~39  &13  &   .517441270E+04 &.84E-08 &.10E-03 &.10E-07
&.25E-11\\
*          &1 &~28   &  ~84  &28  &   .517441270E+04 &.35E-08 &.10E-03 &.10E-07
&.19E-08\\\hline
 {\tt p77} &0 &~15   & ~37   &15  &   .241505129E+00 &.30E-05 &.10E-03 &.35E-04
&.68E-07\\
           &1 &~18   & ~48   &19  &   .241505211E+00 &.61E-04 &.10E-03 &.35E-04
&.14E-05\\\hline
 {\tt p78} &0 &~~9   & ~41   &~9  &$-$.291970041E+01 &.83E-07 &.10E-03 &.15E-05
&.45E-10\\
           &1 &~~8   & ~26   &~8  &$-$.291970041E+01 &.11E-03 &.10E-03 &.15E-05
&.11E-08\\\hline
 {\tt p79} &0 &~~7   & ~24   &~7  &   .974340336E$-$01 &.12E-04 & .10E-03 &.15E
-03&.41E-07\\
           &1 &~10   & ~34   &10  &   .974340336E$-$01 &.66E-05 & .10E-03 &.15E
-03&.40E-07\\\hline
 {\tt p80} &0 & ~66  &  198  & 20 &   .539498478E$-$01 &.25E-08 & .10E-03 &.15E
-07&.25E-12\\
           &1 & ~~7  &  ~21  & ~7 &   .539498478E$-$01 &.91E-08 & .10E-03 &.15E
-07&.11E-07\\\hline
 {\tt p81} &0 & ~59  &   177 & 20 &   .539498478E$-$01 &.55E-05 & .10E-03 &.80E
-06&.36E-09\\
           &1 & ~~8  &   ~24 & ~8 &   .539498419E$-$01 &.63E-05 & .10E-03 &.80E
-06&.17E-06\\\hline
 {\tt p99} & 0& 111  &  269  & 38 &$-$.831079886E+09   &.17E+03 & .10E-03 &.10E
-07&.92E-09  \\
           & 1& 130  & 1229  &130 &$-$.831079886E+09 &.33E+01 & .10E-03 &.10E-0
7&.50E-01  \\\hline
 {\tt p107}&0 & ~16  &  116  & 14 &   .505501180E+04 & .56E-02& .10E-03 &.10E-0
7&.48E-09  \\
           &1 & ~16  &  109  & 16 &   .505501180E+04 &.69E-03 & .10E-03 &.10E-0
7&.39E-09\\\hline
 {\tt p109}&0 &      &       &  * &                  &        & .10E-03 &.10E-0
7&  \\
           &1 &      &       &  * &                  &        & .10E-03 &.10E-0
7&  \\\hline
 {\tt p114}&0 &      &       & *  &                  &        & .10E-02 &.10E-0
3&  \\ 
           &1 &18241 &  941  &924 &$-$.176880696E+04 &.51E-05 & .10E-02 &.10E-0
3&.22E-12  \\\hline
\end{tabular}
}

\nopagebreak
\vspace{0.4em}
\hspace{8em}Table 3: Results for General Problems with FSQP Version 3.2

\end{document}
